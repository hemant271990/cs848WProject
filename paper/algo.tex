This section will give the details of our algorithms.
Our algorithm consists of two main algorithms, first algorithm guarantees that the repair for a given repair expression changes minimal number of cells, i.e. it will be cardinality-minimal.
Second algorithm generates all possible repair expressions, such that their repairs do not introduce new violations.
Together, they guarantee a sample repair within a Set-Minimal space.
For ease of understanding we give details of the second algorithm first.

\subsection{Repair expression generation}
Given a database and a set of DCs, we use Algorithm 1 to carefully generate all possible repair expressions that do not introduce new violations.
It starts by computing the violations and constructing a conflict hypergraph.
The outer loop (lines 3-20) traverse through all the hyperedges of the CH.
For each hyperedge we select one of its "non-frozen" cell (line 5), and construct a repair context and get assignments for it (lines 6-12).
After receiving the assignments we do a look-ahead and check if the assignment is safe and will not cause new violations (line 13).
If it is safe we go ahead and apply the updates and mark the cells of the updated edges as "frozen" (lines 14-16).
If it is not safe we choose another vertex of the current hyperedge, and repeat the process.
The LOOKUP and DETERMINATION procedures are same as described in \cite{XuChu}.
However, we have used a new procedure called, $isSafeAssignment()$ to do a one step look-ahead and determine if the assignment will introduce new violations or not.
Algorithm 2 shows the details of it.
We basically apply the assignments on a copy $data'$ of the actual data and re-create a CH and check for any new hyperedge.

\begin{algorithm}
\caption{Holistic Repair}
\begin{algorithmic}[1] 
\REQUIRE Database Data, Set of Denial Constraints DCs
\ENSURE Repaired Data
\STATE Compute violations, Conflict Hypergraph (CH)
\STATE allEdges $\leftarrow CH(Edges)$
\WHILE{allEdges is not Empty}
\STATE currEdge $\leftarrow$ Get one random edge from allEdges
\WHILE {exists (currCell $\leftarrow$ Get one cell from currEdge) AND currCell.freeze = FALSE}
\STATE rc $\leftarrow$ Initialize a new repair context for currCell
\STATE edges $\leftarrow$ Get all hyperedges for that cell
\WHILE {edges is not empty}
\STATE edge $\leftarrow$ Get an edge from edges
\STATE $LOOKUP(cells,edges,rc)$
\ENDWHILE
\STATE assignments $\leftarrow DETERMINATION(cells,exps)$
\IF {isSafeAssignment(assignments)}
\STATE data.updates(assignments)
\STATE allEdges $\leftarrow$ allEdges $-$ assignments.edges
\STATE $\cup assignment.edges.cells.freeze \leftarrow TRUE$
\STATE Break
\ENDIF
\ENDWHILE
\ENDWHILE
\RETURN data.$PostProcess()$
\end{algorithmic}
\end{algorithm}

\begin{algorithm}
\caption{isSafeAssignment}
\begin{algorithmic}[1] 
\REQUIRE Set of assignments for data
\ENSURE Boolean value, TRUE if assignments do not cause new violations.
\STATE $data'.updates(assignments)$
\STATE rebuild CH for $data'$
\IF {any new hyperedges}
\RETURN FALSE
\ELSE
\RETURN TRUE
\ENDIF
\end{algorithmic}
\end{algorithm}

\subsection{Repairs for Expressions}
This section provide details about determining a valid and minimal assignment for a repair expression.
For this part we borrow the idea of DETERMINATION algorithm in \cite{XuChu}.
The determination algorithm provides two types of minimality when assigning values for a repair expression.

\textbf{Distance minimality.} In this case we minimize the distance given by the following function:
\begin{displaymath}
\sum\limits_{t \in I, t' \in I_r, A \in A^R} dis_A(I(t[A]),I(t'[A]))
\end{displaymath}
We use a squared distance function in our approach.
In particular, we solve the following objective function:
for each cell with value $v$ involved in a predicate of the DC, a variable $x$ is added to the function with $(x - v)^2$.
As the given objective function is quadratic, it has a positive definite matrix, and can be efficiently solved by the quadratic programing technique.

\textbf{Cardinality Minimality.} In this case we want (i) to minimize the number of changed cells, and (ii) to minimize the distance for those changing cells.
In order to achieve cardinality minimality we gradually increase the number of cells that can be changed until the quadratic equation becomes solvable.
For those variables we decide not to change, we add constraint to enforce it to be equal to original value.
It can be seen that this process is exponential in the number of cells in the repair context.

We use one of the above two minimality while generating an assignment for a repair expression.
Say, we always use the Distance Minimality for any expression, now from our Algorithm 1 we generate all possible expressions which do not cause new violations.
Our claim is that all the possible repairs we generate are Distance Minimal with respect to their expression and none of the repairs are Not-Set-Minimal.
